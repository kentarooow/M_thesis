\documentclass[../main]{subfiles}

\graphicspath{{../figures/}}

\begin{document}

\section{正常音の空間マッピング}
\label{sec:pmethod_mapping}

提案手法では,点検作業を行うロボットの自己位置の入力に応じて,正常音を予測するモデルを学習する.
これを目的として,異常音が点検対象の区域に存在しないことを確認したのち,ロボットは,正常音とその位置情報を記録しながら,
点検区域を走行し,訓練データを収集する.
この節では,まず,各地点における正常音が一意に決定されることを示し,
次にロボットの自己位置を入力とし,各地点における正常音を予測する空間マッピング関数のモデル化について述べる.
\subsection{音源固定時の音一意性}


まず,モーターを含む回転機械が一定の動作条件下にある場合,得られる音響信号は時間軸上で有意な変動がなく,
音響パワースペクトル密度が長時間にわたり一定の傾向を示すことが知られている\cite{beranek1992noise}.
本研究で対象とするプラント環境においても,主な音源はポンプであり,ポンプから発生する音響信号も回転機器由来の信号である.
そのため稼働状況が変化しない場合,それぞれの機器からの音響信号も定常的な特徴を持つ.

次に,複数の回転機器が固定された状態において,各観測点で取得される音響信号を考える.
まず,音源が固定された状態において,ある観測点を \(\mathbf{p}\) とする.
音源の位置を \(\mathbf{q}\) とし,音源から発せられる音響信号が定常的であるため,時間に対して不変な値を \(\mathbf{s}\) とみなす.
ここで,観測点 \(\mathbf{p}\) と音源 \(\mathbf{q}\) との距離を \(d(\mathbf{p}, \mathbf{q})\) とし,音の減衰が距離に応じて決まることを考慮し,
距離に応じてどれだけ音が減衰するかを表す係数を \(a\bigl(d(\mathbf{p}, \mathbf{q})\bigr)\) と定義する.
このとき,観測点 \(\mathbf{p}\) での音響信号 \(y(\mathbf{p})\) は以下のように表される.

\begin{equation}
  y(\mathbf{p}) = a\bigl(d(\mathbf{p}, \mathbf{q})\bigr) \, \mathbf{s}.
\end{equation}

この関係式により,音源位置 \(\mathbf{q}\) が固定されている場合,各観測点 \(\mathbf{p}\) における音響信号は一意に定まる.

さらに,複数の音源 \(\mathbf{q}_i\)(\(i=1,\dots,N\))からの音が同時に重なる場合,
音源 \(i\) から発せられる定常的な音響信号を \(\mathbf{s_i}\) とすると,観測点 \(\mathbf{p}\) での総受信音信号 \(Y(\mathbf{p})\) は以下のように表される.

\begin{equation}
  Y(\mathbf{p}) = \sum_{i=1}^{N} a\bigl(d(\mathbf{p}, \mathbf{q}_i)\bigr) \, s_i.
\end{equation}


以上により,複数の音源が存在しても,各観測点 \(\mathbf{p}\) における音響信号は距離関数と音源信号から一意に決定される.


\subsection{空間マッピング関数のモデル化}
本研究では正常音の空間マッピングを実現するために,ロボットの自己位置を入力とし,
各地点における正常音を予測する関数をモデル化する.
空間マッピング関数 \( f \) は、ロボットの自己位置 \( \mathbf{p} = (x, y) \) を入力とし、その地点における正常音の特徴量 \( \mathbf{s} \) を出力する関数として定義される。具体的には、以下のように表される。

\begin{equation}
  \mathbf{s} = f(\mathbf{p}) = f(x, y).
\end{equation}

次の節では,この空間マッピング関数のモデルの出力の選定や,その学習方法について述べる.


\end{document}
