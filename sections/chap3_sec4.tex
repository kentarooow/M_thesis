\documentclass[../main]{subfiles}

\graphicspath{{../figures/}}

\begin{document}

\section{正常音の空間マッピング}
\label{sec:pmethod_mapping}

提案手法では,点検作業を行うロボットの自己位置の入力に応じて,正常音を予測するモデルを学習する.
これを目的として,異常音が点検対象の区域に存在しないことを確認したのち,ロボットは,正常音とその位置情報を記録しながら,
点検区域を走行し,訓練データを収集する.
この節では,まず,各地点における正常音が一意に決定されることを示し,
次にロボットの自己位置を入力とし,各地点における正常音を予測する空間マッピング関数のモデル化について述べる.
\subsection{音源固定時の音一意性}

モーターを含む回転機械が一定の動作条件下にある場合,得られる音響信号は時間軸上で有意な変動がなく,
音響パワースペクトル密度が長時間にわたり一定の傾向を示すことが知られている\cite{beranek1992noise}.

次に音源が固定された状態において,地点Aで聞こえる音を数式であらわす.
音源から地点Aまでの距離をdとし,音源から発せられる音信号をs(t)とする.
地点Aでの受信音信号\(\hat{s}(t)\)は,以下のようにあらわされる.

\begin{equation} y_A(t) = \frac{1}{d^\alpha} s(t - \tau). \end{equation}

ここで,αは音の減衰に関連する減衰係数,τは音源から地点Aまでの伝播時間である.
音源からの音は距離dに応じて減衰し,時間遅延τを伴って地点Aに到達する.
この関係式により,音源位置が固定されている場合,各地点における音響信号は一意敵に決定される.

更に,複数の音源からの音が地点Aで重なり合う場合,それぞれの音源からの音信号が重畳される.
複数の音源からの受信音信号を とすると,地点Aでの総受信音信号は以下のように表される.

\begin{equation} Y_A(t) = \sum_{i=1}^{N} y_{A,i}(t) = \sum_{i=1}^{N} \frac{1}{d_i^\alpha} s_i(t - \tau_i). \end{equation}

ここでNは音源の数,\(d_i\)は各音源から地点Aまでの距離,\(s_i(t)\)は各音源からの音信号,\(\tau_i\)は各音源から地点Aまでの伝播時間である.
各音源からの音信号 \(s_i(t)\)が定常的な特徴を持つ場合,それらの重ね合わせ
\(Y_A(t)\)は,
\begin{equation} Y_A(t) = \sum_{i=1}^{N} \frac{1}{d_i^\alpha} s_i(t). \end{equation}

として表され,定常的な音響信号となる.
定常的な信号は,時間に対して不変であるため,音源の位置が固定されており,稼働状況が変化しない場合,
各地点での音響信号は一意に決定される.

\subsection{空間マッピング関数のモデル化}
本研究では正常音の空間マッピングを実現するために,ロボットの自己位置を入力とし,
各地点における正常音を予測する関数をモデル化する.
空間マッピング関数 \( f \) は、ロボットの自己位置 \( \mathbf{p} = (x, y) \) を入力とし、その地点における正常音の特性 \( \mathbf{s} \) を出力する関数として定義される。具体的には、以下のように表される。

\begin{equation}
  \mathbf{s} = f(\mathbf{p}) = f(x, y).
\end{equation}

次の節では,この空間マッピング関数のモデルをどのような構造を用いて,学習するための手法について述べる.


\end{document}
