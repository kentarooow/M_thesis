\documentclass[../main]{subfiles}
\begin{document}

\chapter*{謝辞}
\label{thankyou}
\addcontentsline{toc}{chapter}{謝辞} % 目次に謝辞項を追加する

\lhead[謝辞]{}
\thispagestyle{empty}

\newpage

% textlint-disable ja-technical-writing/no-mix-dearu-desumasu

本論文を締めくくるにあたり,ご指導ご協力頂きましたすべての皆様に深く感謝申し上げます.

東京大学大学院 新領域創成科学研究科 人間環境学専攻 山下淳先生には2年間で数多くのアドバイスをいただきました.
グループミーティングでは,スライドのわかりづらいポイントや適切に議論を交わすために必要なスライド修正の指摘を頂き,
またプレゼン資料の流れの中で考えを詰め切れていない部分に対する具体的なアドバイスをいただきました.
これらを通じて,研究の深堀や論理的な整理を進めることができ,大変感謝しております.
また,留学をしたい意志を伝えた際には,応募書類の訂正から修士修了に向けた計画まで,
親身になってご相談に乗っていただきました.
その後支援のおかげで実現した留学では,蛇型ロボットやソフトロボティクスといった新しいテーマに触れることができただけでなく,
英語を用いた議論を通じて研究における科学的なアプローチを学ぶ貴重な経験となり,これらの経験は,本修士論文の執筆にも活用することができました.
これらの経験は,研究だけでなく,今後取り組む様々な課題や問題を解決する際にも大いに役立つと確信しております.
このような成長の機会を与えてくださったことに,改めて心より感謝申し上げます.


東京大学大学院 新領域創成科学研究科 人間環境学専攻 安琪先生には,
主に,スライドのわかりやすさの向上や,論理的な展開についてのご指摘を頂きました.
また,留学が決定した際には,先生ご自身の留学経験を踏まえた主に時期の選定に関するアドバイスを頂き,留学の準備において大変参考にさせていただきました.
心より御礼申し上げます.

東京大学大学院 新領域創成科学研究科 人間環境学専攻 濵田裕幸先生には,主にメンタル面でのサポートを多くしていただきました.
廊下ですれ違うたびに,必ず声をかけていただき,短い会話を通じて日常生活の悩みなどについても気軽に相談できる環境を作っていただきました.
更に,実験室へのデスク搬入を手伝ってくださり,実験室での作業の効率性が多いに向上しました.
誠にありがとうございました.

東京大学 東京カレッジ 淺間一先生には,科学的な研究における発表の方法について,多くのご指導を頂きました.
入学当初は,研究の発表においてどのような発表が求められるのかがわからい点が多く,発表資料に多くの不備を残した状態で,ミーティングに臨んでいましたが,
その都度,スライドの細かい部分までご助言を頂きました.
誠にありがとうございました.

東京大学大学院 i-Construction システム学寄付講座特任講師 ルイ笠原純ユネス先生には,
週に1回のペースでミーティングを組んでいただき,そこでは主に研究についての議論を行わせていただいたり,進捗報告を共有させていただきました.
特に,修士1年時には,研究の方向性や進め方において詰まることが多く,その都度具体的なアドバイスをいただきました.
また,提案手法のコンセプトを重視したプレゼン資料作成におけるノウハウや,論文執筆においても,論理的な展開や説明の仕方について多くのご助言をいただきました.
更に柏キャンパスにも頻繁に足を運んでくださり,飲み会などの際には,研究に関する話だけでなく,日常生活の悩みなどに関しても,
積極的に相談に乗ってくださり,精神的な面でも支えていただきました.
心より御礼申し上げます.


東京大学大学院 工学系研究科 人工物工学研究センター学術専門職員 神田真司先生には,
共同研究先である,ENEOS株式会社とのミーティングを毎回設置していただきました.
また,現場にて実験を行う際には,実験を行う日を調整していただいたり,実験の進行においても,具体的なアドバイスをいただきました.
これらのサポートのおかげで,研究に専念し,実験を円滑に進めることができました.
更に,就職活動に関しても,積極的に相談に乗ってくださり,多くの具体的なアドバイスをいただきました.
誠にありがとうございました.

研究室の先輩である,杉浦さんには,研究において多くのサポートを頂きました.
本論文において,チェッカーを引き受けてくださり,論文の完成に向けて多くの助けをいただきました.
また,留学を考えていた際に,背中を押してくださる言葉を頂いたり,人生の進路の選択において,多くのアドバイスを頂きました.
心より感謝申し上げます.



研究室の後輩の正田晃己君には,後輩でありながらも,多くのことを学ばせていただきました.
グループミーティングでの指摘をうまくくみ取れず,自分の考えが行き詰っていた際には,
論理の流れを一から構成しなおすために親身に議論に付き合ってくれました.
そのような献身的なサポートのおかげで,研究が停滞していた時期に新たな道筋を見つけることができ,
気持ちが沈んでいた際にも大きな精神的な支えとなりました.
また,実験を進める際には,持ち前のハードウェアに関する知識を生かし,提案手法を最もシンプルに検証するための実験環境の設定や,
機材の調達において,多くのアドバイスをいただきました.
その結果,データ収集を効率よく進めることができ,研究成果を形にするうえで大きく貢献していただきました.
更に,正田君の豊かな発想力には何度も助けられました.研究の壁に直面するたびに,クリティカルなアイデアを数多く提案してくれたことで,
新たな視点を得られ,課題解決の糸口を見出すことができました.
心より御礼申し上げます.



グループCの皆様には,グループ外でありながらも,多くのサポートを頂きました.
グループCでは「ブリーフィング」と呼ばれる週1回の研究進捗報告会が定期的に開催されており,私はグループ外の立場にもかかわらず,その場に参加する貴重な機会をいただきました.
この機会により,異なる専門分野からの視点を取り入れることが可能となり,研究をより広い視野と客観的な観点で進めることができました.
その結果,研究の質と解像度の両面で大きな成果を上げることができたと感じております.
特に,グループCのリーダーである伊賀上さんには,研究の進め方だけでなく,豊富なご経験に基づいた,研究活動で生じがちなストレスを戦略的に軽減する方法などのドクターならではの知恵を多く教えていただきました.
そのおかげで,より効率的かつ前向きな姿勢で研究に取り組むことができました.
この場を借りて改めて感謝申し上げます.

研究室の同輩には,研究におけるサポートはもちろんのこと,日常生活においても,多くの助けをいただきました.
まず,研究面で頂いたサポートですが,
荻原祐介君には,持ち前のコンピュータビジョンの知識を生かし,実験をするにあたって,ロボットの自己位置の特定が必要になった際に,
ARマーカーを使用した自己位置推定の手法を提案していただきました.
また,山口君には,論文執筆にて必須のツールである,latexのテンプレートを共有していただいたり,使い方を教えていただいたことで,
論文執筆の効率化に大いに貢献していただきました.
また,日常生活においては,研究室の掃除や,研究室の雰囲気づくりにも積極的に参加していただき,研究室全体の雰囲気を良くすることに貢献していただきました.
更に,居室における雑談では,研究に関する話だけでなく,日常生活の悩みや趣味についても話し合い,多くの気づきをいただきました.
誠にありがとうございました.

研究室の後輩には,主に,日常生活において,多く助けられました.
窪田君,森田君,内山君は,ウィンタースポーツやサウナなど,積極的に私の趣味を共有して頂き,これらのアクティビティを共に行うことで,日常生活のストレスを大いに解消することができました.
また,早瀬さんは,コンパ係としての仕事を忙しいタイミングで,代わりに引き受けていただいたり,研究活動の円滑に進めるためのサポートを頂きました.
更に,孫君には,本論文のチェッカーを引き受けていただき,修士論文の完成に向けて,多くの助けをいただきました.
大変ありがとうございました.


秘書の大野朋美さん,小沼博子さんには,物品購入や,研究室でのイベントの手配など,多くの面でサポートをいただきました.
特に研究室にてイベントの運営を行う際には,イベント会場の予約を手伝っていただき,大変助かりました.
心より御礼申し上げます.

共同研究先であるENEOS株式会社の木下将嘉様, 笠原清司様, 甲田梨沙様, 田村直様, 伊藤裕之様, 加藤俊哉様, 野中史彦様, 大和尚也様には,
月一回のミーティングの中で現場の貴重なご意見を頂戴いたしました.
更に,現場実験や工場見学の手配をしていただき,現場に近い環境を具体的に想定することができたため,研究の問題設定において,より具体的な課題を設定することができました.
心より御礼申し上げます.

留学先である,ミュンヘン工科大学のMIMED研究室のメンバーには,留学中の研究において多くのサポートを頂きました.
頻繁に質問や相談に乗ってくださり,科学的な研究における論理的な展開に関するアドバイスを多くいただきました.
留学期間中に学んだそれらの考えは,本修士論文の執筆においても活用することができました.
誠にありがとうございました.

\begin{flushright}
  2025年1月 田中健太郎
\end{flushright}
% textlint-enable ja-technical-writing/no-mix-dearu-desumasu

\end{document}
