\documentclass[../main]{subfiles}

\graphicspath{{../figures/}}

\begin{document}

\section{異常検知手法}
本研究において異常とは,対象とするデータや現象が正常な状態かや挙動から大きく逸脱したパターンやふるまいを指す.
この定義は,計測対象,利用目的,専門家の知見に依存し,金融分野での不正取引検出,医療分野での病変検出など,
様々な領域において応用されている.
次節以降では,まず教師あり学習に基づく異常検知手法を取り上げる.
続いて,異常ラベルが利用困難な場合に適用可能な教師なし学習ベースの手法を紹介し,深層学習を含む近年の動向について言及する.
\subsection{教師あり学習に基づく異常検知}
教師あり学習に基づく異常検知では,あらかじめ正常及び異常クラスに分類されたデータを用いてモデルを学習し,
未知のデータが正常クラスに属するか異常クラスに属するかを予測する.
異常クラスに対して十分なサンプルが収集可能であれば,従来の2値分類問題と同様の手法が適用できるため,高精度な検出モデルを構築しやすい.
具体的な手法として,サポートベクターマシンやランダムフォレスト,ロジスティック回帰といった従来型機械学習手法から,近年ではディープラーニングを活用した
畳み込みニューラルネットワークやリカレントニューラルネットワークを用いたモデルまで,多様な手法が適用可能である.

しかし,教師あり学習を用いる場合,異常クラスのサンプル数が正常クラスに比べて極端に少ない場合,クラス不均衡問題が発生し,
検出性能が低下することが知られている.
前節で述べたように本研究の対象であるポンプなどの回転機器における異常音データも例外ではなく,この問題への対処は極めて重要な課題となっている.

\subsection{教師なし学習に基づく異常検知}
教師なし学習に基づく異常検知は,異常データを十分に得ることが難しい場合に有効である.
教師なし学習では,正常状態の特徴を学習し,その特徴から大きく外れたデータを異常と判断する手法が主に用いられる.
代表的な手法として, 主成分分析やカーネル密度推定,k-means法,One-Class SVMなどが挙げられる.
これらの手法は,データの分布やクラスタ構造をモデル化,もしくは高次元データを低次元空間に射影することで,データの主要なパターンを抽出することによって異常検知を行う.
また,最近では,深層学習を用いた異常検知手法も提案されており,AutoencoderやVariational Autoencoder,Generative Adversarial Networkなどがある.
これらのモデルでは,与えられた訓練データを圧縮及び復元する際,元のデータとの差が最小になることを目的として学習を行い,データの損失を最小限に抑えつつ次元削減を行う.
したがって,訓練後に提供されたデータサンプルが適切に圧縮及び復元されない場合,それは当該データサンプルが訓練データと同じ分布に属していないことを意味する.
そのため,訓練データが正常なデータのみで構成されている場合,そのデータサンプルは異常であると推測することができる.
本研究で対象とする,ポンプなどの回転機器のベアリングの異常音データに対しても,異常のデータの取得が困難であるため,
教師なし学習に基づく異常検知手法を目指す.


\end{document}
