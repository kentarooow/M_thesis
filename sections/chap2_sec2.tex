\documentclass[../main]{subfiles}

\graphicspath{{../figures/}}

\begin{document}

\section{異常検知手法}
\label{sec:related_work_anomaly}
異常検知はタスクの特徴に応じて異なるアプローチがとられる.
本節では,異常検知のタスク設定における分類を述べ,それぞれのタスクに対する異常検知手法を紹介する.
\subsection{教師あり異常検知}
異常検知の代表的なアプローチとして教師あり異常検知(Supervised AD)が挙げられる.
教師あり異常検知では,異常か正常化のラベルが付与されたデータを用いて異常を検出するモデルを学習し,未知のデータが異常か正常かを予測する.
異常事例が明確に収集・定義されている場合に有効であり,異常に属するデータの特徴を直接学習できるため,高い精度を発揮することが多い.

教師あり学習のアプローチを用いた異常検知には,Huらによる研究がある\cite{Hu}.
Huらは,熱画像を異常の検出の対象としており,隣接するピクセル間の相互作用を適応的に再構成し,重要な特徴を強調し,異常を検出する手法を提案した.

また,教師あり異常検知は本研究で対象とする音を用いた異常検知においても有効である.
音を用いた異常検知においても,教師ありアプローチは有効である.
YaoやHarsheyらは,音響信号に対して短時間フーリエ変換で得られたスペクトログラムを入力とし,畳み込みニューラルネットワーク(CNN)\cite{cnn}により異常を検出する手法を提案している\cite{hershey}\cite{yao}.
ただし,教師あり異常検知では,異常サンプルの数が極端に少ない状況においてクラス不均衡問題が発生し,検出性能が低下することが知られている\cite{nilanon}.
\subsection{教師なし異常検知}
教師なし異常検知(Unsupervised AD)は,異常データを学習データとして利用せず,正常データのみを用いてモデルを学習する手法である.
そのため,本研究で対象とするような異常データを事前に十分に集めることができない場合に適している.

教師なし異常検知には大きく特徴埋め込みを用いる方法と,再構成誤差を用いる方法の2つがある.
特徴埋め込みを用いる方法では,正常データの特徴分布を推定し,新たなサンプルがその分布からどれだけ逸脱しているかを測定するアプローチを利用する.
Jihunらはこのアプローチを採用し,画像をパッチ単位に分割し,各パッチがどれほど正常分布から外れているかを評価することで異常を検出する手法を提案している\cite{jihun}.

再構成誤差を用いるアプローチでは,モデルを用いて与えられたデータを圧縮・復元したときの誤差を評価する.
通常の訓練データ(正常データ)の再構成誤差は小さくなるよう学習されるため,訓練データと異なる分布にあるサンプルは誤差が大きくなり,異常と判定できる.
この手法に用いられる代表的なモデルとしては,AutoencoderやGenerative Adversarial Networkなどが挙げられる.
Zavrtanikらは画像の一部を隠して復元するアプローチを導入し,通常のオートエンコーダで問題となる「異常な領域もうまく再現してしまう」課題を解決している\cite{zavrtanik}.
音響信号への応用例としては,Ohらが自動生成ラインに組み込まれるチップマウンターを対象にオートエンコーダで異常音を検出している\cite{oh}.


\subsection{ゼロショット学習}
上記の教師あり/教師なし以外にも,ゼロショット学習が提案されている.
ゼロショット学習では,新しいデータや追加学習を行わず,既存の知識を用いて異常を検出する.
Liらは視覚と言語の統合モデルであるCLIPを活用し,「正常」「異常」といったテキストプロンプトを用いて異常検知を行っている\cite{li}.
ゼロショット学習を用いることで異常データを学習に使用せずに検出できるが,音源が密集している環境では環境ノイズや重複した音源の影響が大きく,適用が難しいことが考えられる.


以上を考慮すると,プラント環境のような音源が密集している環境下かつ,異常音源が取得することが困難である場合には,教師なし異常検知が有効であり,本研究でも教師なし異常検知をアプローチとして用いる.

\end{document}
