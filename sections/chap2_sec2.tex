\documentclass[../main]{subfiles}

\graphicspath{{../figures/}}

\begin{document}

\section{固定マイクを用いた異常音検知}
\label{sec:intro_my_purpose}
特定の環境内における,異常音の検知は,マイクを恒常的に一定の位置に設置し,
日常的に取得される音響信号を用いて異常の有無を判定する手法が中心となっており,
これを本研究では固定マイクを用いた異常音検知と呼ぶ.


固定マイクを用いた異常音検知には,生産環境でのロボットアームの異常音の監視や,産業環境など様々な環境にて
利用されており,実際の環境に合わせて教師あり学習や教師なし学習を用いた手法を適用することが一般的である.

しかし,固定マイクを用いた従来の異常音検知手法は,主に異常が発生したか否かを判定することに特化しており,
異常音源の位置特定に関しては大きな制約が存在する.

本研究で対象とする石油プラント以外にも,音源が密集している環境は存在し,異常音源の位置特定が求められる場面が存在する.
複数のマイクロフォンを用いるマイクアレーを用いて異常音源を推定する手法も提案されているが,この手法では,
異常音源の方向を推定することに特化しており,異常音源の位置特定には至っていない.
そのため,石油プラント内のような音源の密集している空間では,マイクの位置によっては同一の方向に音源が存在する可能性があり,
異常音源の位置特定が困難である.

\end{document}
