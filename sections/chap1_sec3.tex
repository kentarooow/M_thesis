\documentclass[../main]{subfiles}

\graphicspath{{../figures/}}

\begin{document}

\section{本研究の目的}
\label{sec:intro_my_purpose}
\refsec{sec:intro_plant_current}では,石油精製プラントの点検の現状とその課題について述べた.
\refsec{sec:intro_anomaly-detection}では,プラント内音響点検の特徴と課題について述べた.
特にプラント特有の課題として,以下の2点をあげた.

\begin{itemize}
  \item プラント内での異常音の取得が困難であるため正常音のみを学習に用いて異常を検出する必要がある
  \item プラント内には音源となる機器が密集しており,異常音源の位置特定が重要である
\end{itemize}
以上を踏まえて,本研究の目的は以下のように設定する.

\bigskip
\begin{itembox}[c]{目的}
  \centering
  異常音の教師データが不要な異常の位置特定
\end{itembox}
% textlint-enable ja-technical-writing/ja-no-mixed-period

\end{document}
