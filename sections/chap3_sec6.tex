\documentclass[../main]{subfiles}

\graphicspath{{../figures/}}

\begin{document}

\section{モデルの出力に基づく異常音のエネルギー推定と異常の検出}
\label{sec:pmethod_distance_estimation}

本研究では前節で述べたように,モデルが正常音のメルスペクトログラムを出力するように学習を行う.
学習済みモデルが出力するメルスペクトログラムと観測音のメルスペクトログラムを比較することで,異常音源までの距離を推定する.
以下では,まず短時間フーリエ変換によるパワースペクトルの算出方法を示し,次にメルフィルタバンクによる次元削減の手法を述べる.
また,この式を利用し,次の節でメルスペクトログラムが距離の増加に応じてどのように変化するかを示す.

\subsection{メルスペクトログラムの計算}

\subsubsection{短時間フーリエ変換}
音響信号 $x(t)$ に対して,時間領域信号を短いフレーム(ウインドウ)に分割し,各フレームごとにフーリエ変換を行うことで,時間-周波数領域への変換を行う.
各フレームを $x[n]$ として離散化し,フレーム長を $N$,ウインドウ関数を $w[n]$ とすると,フレーム $m$ における短時間フーリエ変換(STFT)は以下のように定義できる.

\begin{equation}
X(m, k) = \sum_{n=0}^{N-1} x[mL + n]\, w[n]\, e^{-j \frac{2\pi k n}{N}}.
\end{equation}

ここで,$m$ はフレームインデックス,$k$ は周波数ビンインデックス,$L$ はフレームシフト幅である.
得られた STFT 係数 $X(m, k)$ からパワースペクトル $P(m, k)$ を求める.
パワースペクトルは STFT 係数の絶対値二乗として定義できる.

\begin{equation}
P(m, k) = |X(m, k)|^2.
\end{equation}

\subsubsection{メルフィルタバンクの適用}
次に,パワースペクトル \(P(n, k)\) にメルフィルタバンクを適用する.
\refsec{sec:pmethod_feature_selection} で述べたように,ニューラルネットワークの出力として用いる特徴量はできる限り次元を削減することが望ましい.
メルフィルタバンクはそのための有効な手法である.

メルフィルタバンクでは,周波数 \(f\) [Hz] をメル尺度
\begin{equation}
M(f) = 2595 \log_{10}\left(1 + \frac{f}{700}\right).
\end{equation}
によって対数的に変換し,メル軸上で等間隔になるよう複数の三角形状フィルタを配置する.
それぞれのフィルタは周波数ビン \(k\) に対して重み付けを行い,出力を加算することで周波数領域の情報を集約する.
フィルタ数を \(R\) としたとき,メルフィルタバンク出力 \(S(n, r)\) は次のように定義する.

\begin{equation}
S(n, r) = \sum_{k} P(n, k) \, H_{r}(k).
\quad (r = 1, 2, \dots, R)
\end{equation}

ここで,\(H_{r}(k)\) は三角形状の重みを与えるメルフィルタであり,\(r\) はフィルタの番号を表す.
この \(S(n, r)\) は元のパワースペクトルより次元が低く,次元削減の効果がある.

一般的なメルスペクトログラムでは対数変換を行うことが多いが,本研究では後述する距離推定特性を利用するため,対数変換を行わずに \(S(n, r)\) を出力とするモデルを考える.

\subsection{異常音エネルギーの推定と異常の検出}
フレーム \(n\) における自己位置 \(p(n)\) を学習済みモデルに入力として与えることで,フレーム \(n\) における正常音のメルスペクトログラム \(S_{\mathrm{normal}}(n, r)\) を取得できる.
ここで,前節で計算したメルスペクトログラムはパワースペクトルをメルフィルタにより集約したものであり,パワースペクトルがエネルギー密度を表すことから,メルスペクトログラムも人間の聴覚特性に基づいたエネルギー密度を表すといえる.
そのため,観測されたメルスペクトログラム \(S_{\mathrm{observed}}(n, r)\) と学習済みモデルが出力する正常音のメルスペクトログラム \(S_{\mathrm{normal}}(n, r)\) の差分を \(L_1\) ノルムで計算することで,フレーム \(n\) における異常音の周波数領域におけるエネルギー総和 \(E(n)\) を算出できる.

\begin{equation}
E(n) = L_1\bigl(S_{\mathrm{observed}}(n, r),\, S_{\mathrm{normal}}(n, r)\bigr).
\end{equation}

異常を検出する際は,このエネルギー \(E(n)\) が閾値を超えたフレーム \(n\) を異常音が存在するフレームとみなす.
また,ここで得られた異常音のエネルギーとロボットの自己位置を組み合わせることで,異常音源の位置を推定する際に利用する.

\end{document}
