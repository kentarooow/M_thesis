\documentclass[../main]{subfiles}

\graphicspath{{../figures/}}

\begin{document}

\section{問題設定}
\label{sec:problem_setting}
まず,本研究で想定されているプラント環境について説明する.
項で述べたように,本研究で扱うプラント環境は検査対象となる音源が密集している環境を想定し,
多種多様な異常音のデータを取得することはできず,それらの事前知識も使用することはできない.

また,プラント環境には正常な状態と異常な状態の2種類が存在し,正常状態の際は,全ての機器が正常に稼働し,
それぞれの機器に応じた音が発生している.
また,異常状態の際は,異常音を発している機器が1つのみ存在し,残りの機器は全て正常音を出している状態と定義する.
また,異常は頻繁に発生するものではなく,ここで用いた経路内に異常音を発している機器が1つのみという仮定は実際のプラント環境においても成り立つものである.

移動ロボットには,プラント内の機器近くを通過する経路が与えられており,ロボットは常にこの経路を通過する.
また,ロボットには無指向性のマイクロホンを搭載している.
更に,正常状態の経路を走行し,そこで得られたデータをモデルの学習に用いることができる.

目的は,ロボットがプラント内を走行する際に,観測された音から異常音を検出し,その異常音が発生している機器の位置を特定することである.

\end{document}
