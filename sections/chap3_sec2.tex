\documentclass[../main]{subfiles}

\graphicspath{{../figures/}}

\begin{document}

\section{提案手法のコンセプト}
人間は慣れ親しんだ環境において異常な音を容易に識別できる.
これは一定の適応期間を経ることで,その場所において期待される音を予測できるようになり,その結果として通常と異なる音を検出する能力が向上するためである.
本研究では,この特性を模倣するアプローチを提案する.

移動ロボットを用い石油精製プラント内を巡回し,音響点検作業を行うことを考える.
また,このロボットはSLAMによる自己位置推定能力を持つと仮定する.
そして,このロボット上に正常音の空間マッピングモデルを構築し,自己位置推定によって割り出されたロボットの位置に応じた予測音を生成する.
この予測音と,同じ位置で記録された実際の音を比較し,観測音が異常音とどの程度異なるのかを各地点で評価する.
ここに,音が距離に応じて,減衰することを考慮し,それらの空間的な特性を用いることで,異常音の位置を特定する.

提案手法の概要は,図 \ref{fig:concept}に示す通りである.

\label{sec:pmethod_concept}

\end{document}
