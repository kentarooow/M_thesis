\documentclass[../main]{subfiles}
\begin{document}

\section{今後の展望}
\label{sec:conc_future}
屋外実験において,正常状態の経路に対しても異常と出力されてしまう偽陽性が多く見られた.
一般に異常検知の分野においては,偽陽性よりも偽陰性が重大であるとされることが多い.
一方で偽陽性が多い場合,運用者は異常と判断された箇所を確認するために多大な労力を費やすことになる.
そのため正常音の予測の精度を高めることで,偽陽性を減らしていくことが今後の課題の1つとして考えられる.
また,本手法では位置推定の段階において異常音源が1つしかないことを前提としている.
稼働中の製油所では異常が極めてまれであるため,この前提は必ずしも厳しい制約になるとは限らないが,異常が頻繁に発生する可能性がある状況では,
複数の異常音源を推定できることも望まれる.

\end{document}
