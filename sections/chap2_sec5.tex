\documentclass[../main]{subfiles}

\graphicspath{{../figures/}}

\begin{document}

\section{本研究の位置付け}
\label{sec:intro_my_purpose}
% 本節では,関連研究に対する本研究の位置付けについて述べる.図 2.1 に関連研究と本研
% 究との関係を示す.
% 2.2 節では,教師あり学習と教師なし学習を用いた手法の違いについて述べた.プラント
% 内では異常音の取得が難しいという点で,プラント内の機器についての異常検知では教師な
% し学習を用いることが適切であることを述べた.しかし,教師なし学習による異常音検知の
% 研究は,機器 1 台の近くにマイクロホンを設置してその機器の異常音を検出するという設定
% で行われている.プラント内での異常検知では様々な場所で異常を検出する必要があり,そ
% のままの設定で適用するのが難しい.
% 2.3 節では,様々な場所での異常音検知を行うため,複数の場所に固定マイクロホンを配
% 置し異常音検知を行っている研究や,実際のプラントでロボットを用いて行った異常音検知
% の研究を行っている研究について述べた.しかし,それらの手法では場所ごとに取得した音
% 響データをうまく扱えていないことに加え,正常音と異常音の分離に単純なアルゴリズムし
% か用いておらず複雑な正常音と異常音の分離が難しいといった問題点が挙げられる.
% 以上に挙げた問題点の解決するため,空間情報を考慮した教師なし学習手法の構築を目
% 指す.




\end{document}
