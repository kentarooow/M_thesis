\documentclass[../main]{subfiles}

\graphicspath{{../figures/}}

\begin{document}

\section{先行研究}
\label{sec:intro_previous-research}
一般に,異常検知の手法は主に,正常クラスと異常クラスの2クラスへの分類問題として定義され,
これは通常教師あり学習を用いて解かれることが多い.
教師あり学習では,正常クラスト異常クラスのデータがあり,これらのサンプルが正しくラベル付けされている場合,正常クラスと異常クラスの境界を学習することで,
未知のデータが正常クラスに属するか異常クラスに属するかを予測することができる.
% 教師あり学習の代表的な手法として,サポートベクターマシン(SVM)\cite{cortes1995support}や決定木\cite{quinlan1986induction},ランダムフォレスト\cite{breiman2001random},ニューラルネットワーク\cite{lecun2015deep}などがあり,
異常の対象が音である場合には, や の研究がある.
これらの手法は一般に高い性能を示すことが知られている.
更に,あらかじめ,異常音のデータが十分に取得されている場合,観測音と異常音のマッチングを行うことで,
異常音の発生源を特定することが可能であり,プラントの音響点検において音源となる機器が密集しているという課題を解決することができる.

しかしながら,これらの正常データを用いる教師あり学習による異常検知手法は訓練データとして収集されたデータの量と質に依存している.
特に,異常データの量が正常データの量に比べて非常に少ない訓練データを用いて学習する場合,クラス不均衡問題が発生し,異常検知の性能が低下することが知られている.
この,異常データの量が正常データの量に比べて少ないという問題は,実世界の多くの問題において発生し,
本研究で音響点検の対象とするポンプなどの回転機器のベアリングの異常音のデータも例外ではない.
前節で述べたように,異常音はベアリングの傷の深さや種類に依存して異なる性質を持ち,異なる異常音のデータが十分に取得することができない.
また,異常の発生に関しても頻繁に発生するわけではなく,これらの多岐にわたる異常音を十分に取得することは困難である.
このような問題を解決するため,教師なし学習を用いた異常検知手法が提案されている.

教師なし学習は,正常データのみを用いて学習を行い,未知のデータが正常クラスに属するか異常クラスに属するかを予測する手法である.
% 教師なし学習の代表的な手法として,古典的な手法では,主成分分析(PCA)\cite{jolliffe2011principal}やカーネル密度推定\cite{parzen1962estimation},k-means法\cite{macqueen1967some},One-Class SVM\cite{scholkopf2001estimating}などがある.
これらの手法は,データの分布やクラスタ構造をモデル化,もしくは高次元データを低次元空間に射影することで,データの主要なパターンを抽出することによって異常検知を行う.
% また,最近では,深層学習を用いた異常検知手法も提案されており,Autoencoder\cite{hinton2006reducing}やVariational Autoencoder\cite{kingma2013auto},Generative Adversarial Network\cite{goodfellow2014generative}などがある.
これらのモデルでは,与えられた訓練データを圧縮及び復元する際,元のデータとの差が最小になることを目的として学習を行い,データの損失を最小限に抑えつつ次元削減を行う.
したがって,訓練後に提供されたデータサンプルが適切に圧縮及び復元されない場合,それは当該データサンプルが訓練データと同じ分布に属していないことを意味する.
そのため,訓練データが正常なデータのみで構成されている場合,そのデータサンプルは異常であると推測することができる.

しかしながら,オートエンコーダはシンプルな構造ゆえに,複雑なデータの表現を学習することが難しいという問題がある.
らによる研究では,点検対象の区域をグリッドに分割し,それぞれのグリッドごとにモデルを学習することによって,この課題を解決する手法が提案されている.
この手法により,それぞれのモデルが学習するデータの複雑さや種類を大幅に削減することにつながり,オートエンコーダの性能を向上させることができる.

しかしながら,この手法では,グリッドの大きさよりも詳細な解像度で,異常の位置を特定することはできない.
前節で述べたように,プラント内には異常の音源となりうる機器が密集しており,観測した異常音がどの機器によるものなのかということが明確になりにくい.
そのため,プラント内の音響点検では異常の有無の判別だけでなく,異常音の発生源を特定することが非常に重要である.

\end{document}
