\documentclass[../main]{subfiles}
\begin{document}

\section{結論}
\label{sec:conc_conclusion}
本研究では,「異常音の教師データを必要としない,異常音の検知と位置推定」という目的を達成するため,
移動ロボットに搭載可能な正常音の空間マッピングに基づく異常音検知手法を提案した.

第1章では,本研究の背景について述べた.
まず,石油精製プラントの点検の現状として,現場作業員による点検の課題について述べ,点検の自動化の重要性について述べた.
次に,石油精製プラント内で起こる様々な異常とその検知手法を紹介し,その中でも音響信号を用いた異常検知手法が重要であることを述べた.
更に,石油精製プラント内では,異常音の取得が困難であること,そして,異常音の位置推定が重要であることを述べ,
この課題を解決するため,本研究の目的を「異常音の教師データを必要としない,異常音の検知と位置推定」と設定した.

第2章では,プラント内での異常音の検知と位置推定を実現するため関連研究として,異常検知における教師あり学習と教師なし学習を紹介し,
本研究では,異常音の教師データを取得することができないため,タスクとしてより困難である教師なし学習を用いざるを得ないことを述べた.
また,次に移動ロボットを用いた異常音検知の先行研究を紹介した.
先行研究では,異常音の抽出にヒューリスティックに設計したフィルタを用いており,多種多様な異常音の
抽出が困難であること,
また,空間的に離散的なモデルを使用しているため,異常音の位置推定に限界があることを述べた.

第3章では,本研究の提案手法である位置推定が可能な異常音検知手法について述べた.
第2章で述べた,先行研究の課題を解決するため,異常を正常音との予測誤差として捉え,異常音の大きさを用いて位置推定を行う手法を提案した.
その際,プラント内での回転機器などは,音の特徴が定常的であるという仮定を置き,その仮定の下,各地点での正常音の大きさは一定であるため,
正常音を空間的にマッピングすることが可能であり,その予測誤差である異常音の大きさを用いて位置推定を行うことが可能であると述べた.

第4章では,提案手法の有効性を検証するため,本研究で行った実験について述べた.
まず,プラント内環境を模した屋内環境での実験を行い,正常音のマッピングとそれに基づく異常音の検知,位置推定が可能であることを示した.
次に,本研究で対象とするプラント環境にて実験を行い,提案手法が異常音の検知に有効であることを示した.

以上から,本論文により,異常音の教師データを必要としない,異常音の検知と位置推定を実現する手法を提案し,その有効性を示した.

\end{document}
