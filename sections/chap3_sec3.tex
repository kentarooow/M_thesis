\documentclass[../main]{subfiles}

\graphicspath{{../figures/}}

\begin{document}

\section{空間マッピングの学習}

提案手法では,点検作業を行うロボットの自己位置の入力に応じて,正常音を予測するモデルを学習する.
これを目的として,異常音が点検対象の区域に存在しないことを確認したのち,ロボットは,正常音とその位置情報を記録しながら,
点検区域を走行し,訓練データを収集する.
\subsection{前処理}
収集した正常な音データはセグメントに分割され,それぞれのセグメントが訓練データのサンプルに対応する.
音響データは時間ずれの影響を受けやすいため,元の時系列データから静的な特徴量を抽出する必要がある.
一般的に用いられる,短時間フーリエ変換を用いて,周波数領域に変換したのち,メルフィルタバンクを用いて,次元を削減したものを特徴量として用いる.

最終的にデータのスケーリングがMin Maxスケーリングを用いて行われる.
このスケーリングは,データの最小値を0,最大値を1に変換することで,データの範囲を統一するために行われる.

\subsubsection{正常音マップの学習}
ロボットの自己位置座標と音響データの関係を学習するために,ニューラルネットワークを用いる.
ニューラルネットワークは,近年複雑な関係性を認識するために広く用いられている手法であり,
本研究では,このモデルは,入力としてロボットの自己位置座標を受け取り,その座標に対応する音響データを出力とする.

元の音響データをサンプルに分割することは,サンプル同士が独立してしまうという問題が生じる.モデルには,サンプル同士の関係性はないためである.
しかしながら,実世界における音は,劇的に変わるということはまずなく,隣接するサンプルには,連続性がある.
この連続性を反映させるために,Sharpness-Aware Minimizationを用いる.
この手法は,モデル内の重みの更新時に,重みの変化に対する損失関数の変化が小さくなるように重みの更新を行うことで,汎化性能を向上させることを目的とした最適化手法である.
この手法を用いることで,予測された音響データが入力として与えられる座標の変化に応じてスムーズに変化するように学習される.

\label{sec:pmethod_mapping}

\end{document}
