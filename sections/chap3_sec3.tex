\documentclass[../main]{subfiles}

\graphicspath{{../figures/}}

\begin{document}

\section{屋外実験}
\label{sec:vexp_cone-index}
研究[13]において得られた貴重な機会を利用し、実際の石油精製施設内で提案手法の実フィールド条件での性能を評価するためにデータを収集した。この実験はENEOS株式会社が運営する川崎工場で実施され、実験には実験室で使用されたのと同じ機器と前処理が適用された。

図6に、データが収集されたエリアとその経路の概略を示す。
このエリアには、連続的なハミング音を発するポンプや間欠的に圧力放出音を発するバルブなど、複数の音源が含まれている。
工場が実験中も通常稼働していたため、これらに加えて複数の音源が存在し、複雑で高い環境騒音が形成されていた。
ポンプなどの回転または往復運動を行う機械における異常を再現するために、周期的活動を持つ音源が配置された。
具体的には、連続的な衝撃を加えるインパクトドライバーによる音や、1分間に240回の速度で金属棒を叩く音が異常音源として使用された。
この2種類の異常音源は、図中の位置AとBに配置され、実験での異常検知対象となるものとして設定された。
しかしながら、その日の物流の制約により、以下の制約が存在した。

\subsection{実験環境}
\label{subsec:vexp_ci_environment}

\subsection{実験方法}
\label{subsec:vexp_ci_method}
3回の走行では異常音源を設置せず、2回の走行では2種類の異常音源を設置し、それらの音源の位置を2回の走行間で切り替えた。
結果として、6902の正常サンプルと4628の異常サンプルが得られた。2回の正常な走行データは正常音モデルの訓練に使用され、1回の正常な走行データと2回の異常な走行データはテストに使用された。

\subsection{データの処理}
\label{subsec:vexp_ci_processing}

\subsection{実験結果}
\label{subsec:vexp_ci_result}
実フィールド条件における提案手法を用いた異常音検知の結果を図6に示す。

図6(a)では、環境内に異常音が存在しない場合、ほとんどのサンプルが正常として正しく分類されていることがわかる。
図6(b)および図6(c)では、異常サンプルが、異常音源の位置に対応する経路領域においてほぼ正しく分類されていることがわかる。
異常として分類されたサンプルの広がりは、2種類の異常音源の特性および異常音源の位置をA地点とB地点で切り替えた結果を反映しており、この様子は図6(b)と図6(c)の間で明確に確認できる。
インパクトドライバーは連続的な音を発しており、その結果、その周辺ではサンプルが一貫して異常として分類されている。
一方で、金属棒を叩く音はより断続的な検出パターンを生じさせている。
この理由は、叩く音がサンプルで捕捉された時間窓に対応しているためであると推測される。
\subsection{考察}
\end{document}
