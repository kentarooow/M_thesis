\documentclass[../main]{subfiles}

\graphicspath{{../figures/}}

\begin{document}

\section{ニューラルネットワークを用いたモデルの設計}
\label{sec:pmethod_neural_network}

本研究では,座標に対する正常音の予測を行うために,座標を入力としそれに対応する正常音を出力とする関数を
ニューラルネットワークを用いて近似する.
本節では,ニューラルネットワークの基礎について述べた後,モデルの入力と出力,モデルのアーキテクチャについて述べる.
\subsection{ニューラルネットワークの基礎}
ニューラルネットワークは,脳の神経細胞の働きを模倣した数学モデルであり,複数の層から構成される.
各層は複数のニューロンから構成され,前の層のニューロンからの入力を受け取り,それらの入力に対して重み付けを行い,
活性化関数によって出力を計算する.
ニューラルネットワークは,これらの層を積み重ねることで,複雑な関係性を表現することができる.
また,その関係性を学習するために,誤差逆伝播法と呼ばれるアルゴリズムを用いて,重みの更新を行う.
本研究では,正常状態の経路を走行し,得られたデータを用いてニューラルネットワークを学習させることで,
座標に対する正常音の予測を行う.

\subsection{出力に用いる特徴量の選定}
\label{sec:pmethod_feature_selection}

本研究では,ニューラルネットワークを用いて音の空間マッピングを行い,さらにエネルギー情報を利用して異常の位置を推定する.

マッピングを行うため,音の情報を周波数領域への変換によって定常的な特徴を抽出することが必要となる. また,ニューラルネットワークは画像のような高次元のデータを入力とし,低次元のデータを出力することを得意とする. したがって,出力である正常音はできる限り低次元の特徴量を用いることが望ましい. さらに,エネルギー情報を利用して異常の位置を推定する関係上,出力にはエネルギー情報を含む特徴量を用いる必要がある.

周波数領域への変換によって得られる代表的な特徴量として,スペクトログラム,メルスペクトログラム,MFCCなどが挙げられる. スペクトログラムは,音圧波形に短時間フーリエ変換(STFT)を適用して得られる周波数領域の情報であり,幅広い周波数帯域での振幅(エネルギー)を比較的詳細に保持できる. 一方で,周波数解像度が高いことに伴って次元数が大きくなるため,ニューラルネットワークの出力として直接扱うには負荷が大きいという問題がある.

メルスペクトログラムは,スペクトログラムにメルフィルタバンクを適用し,人間の聴覚特性に基づいた周波数変換を行ったものである. スペクトログラムよりも周波数分解能が低い反面,次元数を抑えることができる利点がある. さらに,対数変換後の振幅情報が保持されるため,特徴量間におけるエネルギー情報の比較が可能である.

Mel-Frequency Cepstral Coefficients(MFCC) は,メルスペクトログラムに離散コサイン変換(DCT)を適用し,更に次元削減を行った特徴量である. 低次元化によって学習や識別の負荷を減らせる一方で,対数変換やDCTの過程でエネルギー情報を大幅に簡略化してしまうため,特徴量間におけるエネルギー情報の比較が困難である.


以上のようにスペクトログラムはエネルギー情報を正確に保持できる一方,次元数が大きく,ニューラルネットワークでの出力として扱うには負荷が高い. MFCCは次元数が小さいものの,エネルギー情報を含む振幅情報が大きく失われるため,異常位置の検出精度が低下する可能性がある. 一方,メルスペクトログラムはスペクトログラムよりも次元数を抑えつつ,MFCCほどエネルギー情報を失わないという観点でバランスが良い.

これらの理由から,本研究では,ニューラルネットワークの出力としてメルスペクトログラムを用いる. これにより,周波数領域での定常性を確保しつつ,エネルギーに関する情報をある程度保持し,かつ過度に高次元ではない特徴量を得ることができるため,音の空間マッピングと異常位置推定の両面で有効と考えられる.


\subsection{正常音と座標の連続的な対応関係の学習}
\label{sec:pmethod_sequential}
\refsec{sec:pmethod_mapping}で述べたように,ある環境内の音源が一定の条件下で作動しているとき,その音源から発せられる音は定常的な特徴を示し,
各座標における音の特徴は一定である.
また,そのような条件下では,座標の変化に対して,正常音も連続的に変化すると考えられる.
このような座標と音の連続的な対応関係を学習することによって,正常音の予測がより正確に行えると考えられる.
本研究ではモデルの重みの更新にSAM(Sharpness-Aware-Minimization)アルゴリズムを採用することで,この連続的な対応関係を学習する\cite{sam}.
SAMアルゴリズムは,モデル内の重みの更新時に,重みの変化に対する損失関数の変化が小さくなるように重みの更新を
行うことで,汎化性能を向上させることを目的とした最適化手法である.
このアルゴリズムを用いてモデル内の重みを更新することで,モデルの出力である正常音のメルスペクトログラムが,座標に対して連続的に変化するように学習する.

\end{document}
