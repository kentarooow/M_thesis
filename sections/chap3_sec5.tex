\documentclass[../main]{subfiles}

\graphicspath{{../figures/}}

\begin{document}

\section{ニューラルネットワークを用いたモデルの設計}
本研究では,座標に対する正常音の予測を行うために,座標を入力としそれに対応する正常音を出力とする関数を
ニューラルネットワークを用いて近似する.
本節では,ニューラルネットワークの基礎について述べた後,モデルの入力と出力,モデルのアーキテクチャについて述べる.
\subsection{ニューラルネットワークの基礎}
ニューラルネットワークは,脳の神経細胞の働きを模倣した数学モデルであり,複数の層から構成される.
各層は複数のニューロンから構成され,前の層のニューロンからの入力を受け取り,それらの入力に対して重み付けを行い,
活性化関数によって出力を計算する.
ニューラルネットワークは,これらの層を積み重ねることで,複雑な関係性を表現することができる.
また,その関係性を学習するために,誤差逆伝播法と呼ばれるアルゴリズムを用いて,重みの更新を行う.
本研究では,正常状態の経路を走行し,得られたデータを用いてニューラルネットワークを学習させることで,
座標に対する正常音の予測を行う.
\subsection{モデルの入力と出力}
モデルの入力と出力にはそれぞれ,ロボットの自己位置座標とその座標に対応する正常音のメルスペクトログラムを用いる.
モデルの出力にメルスペクトログラムを用いる理由として,ニューラルネットワークは次元の削減の方が得意であり,
デコーダのようなモデルの構造は,低次元の特徴量から高次元の特徴量を予測するため,そのような高次元の特徴量には,
一般的にノイズが乗りやすいため,メルスペクトログラムを用いて最大限音の次元を削減しつつ,
エネルギーの情報を残しているという理由である.
\subsection{モデルのアーキテクチャ}
モデルの重みの更新にSAM optimizer を用いる.
SAM optimizer は
\label{sec:pmethod_preprocessing}

\end{document}
