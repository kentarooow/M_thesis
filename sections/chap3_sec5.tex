\documentclass[../main]{subfiles}

\graphicspath{{../figures/}}

\begin{document}

\section{ニューラルネットワークを用いたモデルの設計}
\label{sec:pmethod_neural_network}

本研究では,座標に対する正常音の予測を行うために,座標を入力としそれに対応する正常音を出力とする関数を
ニューラルネットワークを用いて近似する.
本節では,ニューラルネットワークの基礎について述べた後,モデルの入力と出力,モデルのアーキテクチャについて述べる.
\subsection{ニューラルネットワークの基礎}
ニューラルネットワークは,脳の神経細胞の働きを模倣した数学モデルであり,複数の層から構成される.
各層は複数のニューロンから構成され,前の層のニューロンからの入力を受け取り,それらの入力に対して重み付けを行い,
活性化関数によって出力を計算する.
ニューラルネットワークは,これらの層を積み重ねることで,複雑な関係性を表現することができる.
また,その関係性を学習するために,誤差逆伝播法と呼ばれるアルゴリズムを用いて,重みの更新を行う.
本研究では,正常状態の経路を走行し,得られたデータを用いてニューラルネットワークを学習させることで,
座標に対する正常音の予測を行う.
\subsection{モデルの入力と出力}
モデルの入力と出力にはそれぞれ,ロボットの自己位置座標とその座標に対応する正常音のメルスペクトログラムを用いる.
モデルの出力にメルスペクトログラムを用いる理由として,ニューラルネットワークは次元の削減の方が得意であり,
デコーダのようなモデルの構造は,低次元の特徴量から高次元の特徴量を予測するため,そのような高次元の特徴量には,
一般的にノイズが乗りやすいため,メルスペクトログラムを用いて最大限音の次元を削減しつつ,
エネルギーの情報を残しているという理由である.
\subsection{正常音と座標の連続的な対応関係の学習}
\refsec{sec:pmethod_mapping}で述べたように,ある環境内の音源が一定の条件下で作動しているとき,その音源から発せられる音は定常的な特徴を示し,
各座標における音の特徴は一定である.
また,そのような条件下では,座標の変化に対して,正常音も連続的に変化することが,式 からわかる.
このような座標と音の連続的な対応関係を学習することによって,正常音の予測がより正確に行えると考えられる.
本研究ではモデルの重みの更新にSAMアルゴリズムを採用することで,この連続的な対応関係を学習する.
SAMアルゴリズムは,Sharpness-Aware-Minimizationの略であり,モデル内の重みの更新時に,重みの変化に対する損失関数の変化が小さくなるように重みの更新を
行うことで,汎化性能を向上させることを目的とした最適化手法である.
このアルゴリズムを用いてモデル内の重みを更新することで,モデルの出力である正常音のメルスペクトログラムが,座標に対して連続的に変化するように学習する.

\end{document}
