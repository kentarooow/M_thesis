\documentclass[../main]{subfiles}

\graphicspath{{../figures/}}

\begin{document}

\section{異常検知}
\label{sec:pmethod_anomaly_detection}

式 (1) の左辺は観測された音と予測された音のエネルギー差に対応する。

音波が運ぶエネルギーは距離の平方根に反比例して減衰することが知られている。すなわち、距離の二乗に反比例する。これは式 (2) で表される。ここで、$E$ は波が運ぶエネルギー、$r$ は観測点と音源との距離である。

\begin{equation}
    E \propto \frac{\alpha}{r^2} \tag{2}
\end{equation}

異常音源の位置 $\mathbf{x}_a$ を定数とし、座標系における適切な距離関数を $d(., .)$ とすると、式 (2) は集合 $S$ のサンプルに対して式 (3) に書き換えられる。

\begin{equation}
    d(\mathbf{x}_a, \mathbf{x}_i^*) = \frac{\alpha}{\sqrt{E_i^*}} \tag{3}
\end{equation}

これが集合 $S$ 内のすべてのサンプルに対して成り立つため、式 (4) を導ける。

\begin{equation}
    \sum_{S} \left| d(\mathbf{x}_a, \mathbf{x}_i^*) - \frac{\alpha}{\sqrt{E_i^*}} \right| = 0 \tag{4}
\end{equation}

ここで、環境内に単一の異常音源のみが存在し、$\alpha$ が環境に依存する定数であると仮定すると、異常音源の位置を求めることが可能になる。

\end{document}
