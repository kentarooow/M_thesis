\documentclass[../main]{subfiles}

\graphicspath{{../figures/}}

\begin{document}

\section{異常音の検出}
正常音のデータによってモデルを学習した後,実際の運用が始まる.
ロボットは環境を巡回する際に,マイクロフォンを用いて音響データが対応する自己位置座標と同時に,記録される.
前節で説明した同様の前処理が音響データに適用され,観測された環境のメルスペクトログラムのセットが得られる.
これをロボットの位置 $\mathbf{x}_i$ に対応する $m_i^o$ とする。
一方で、位置情報は正常データで訓練されたニューラルネットワークに入力され、ロボット位置 $\mathbf{x}_i$ に対して予測された音 $m_i^p$ が得られる。

最後に、観測されたMelスペクトログラムと予測されたMelスペクトログラムの比較が $L_1$ ノルムを使用して行われ、エネルギー差が算出される。これは式 (1) のように表され、異常検知の閾値を $T_h$ とする。

\begin{equation}
    E_i = L_1(m_i^o, m_i^p) < T_h \quad (1)
\end{equation}

予測を生成したモデルは正常データのみを用いて訓練されているため、その予測は環境の正常な状態に対応する。観測された音が予測から逸脱した場合、それは異常音とみなされ、異常音が検出されたロボット位置とその対応するエネルギー差からなる集合
\[
S = \{ (\mathbf{x}_i^*, E_i^*) \}
\]
が得られる。

\label{sec:pmethod_preprocessing}

\end{document}
