\documentclass[../main]{subfiles}

\graphicspath{{../figures/}}

\begin{document}

\section{前処理}
収集した正常な音データはセグメントに分割され,それぞれのセグメントが訓練データのサンプルに対応する.
音響データは時間ずれの影響を受けやすいため,元の時系列データから静的な特徴量を抽出する必要がある.
一般的に用いられる,短時間フーリエ変換を用いて,周波数領域に変換したのち,メルフィルタバンクを用いて,次元を削減したものを特徴量として用いる.

最終的にデータのスケーリングがMin Maxスケーリングを用いて行われる.
このスケーリングは,データの最小値を0,最大値を1に変換することで,データの範囲を統一するために行われる.
\label{sec:pmethod_preprocessing}

\end{document}
