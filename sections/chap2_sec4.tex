\documentclass[../main]{subfiles}

\graphicspath{{../figures/}}

\begin{document}

\section{本研究の位置付け}
\label{sec:related_work_summary}

本節では,関連研究に対する本研究の位置づけについて述べる.
% \reftab{tab:comparison}に関連研究と本研究との関係を示す.
\refsec{sec:related_work_anomaly}では,教師あり学習と教師なし学習を用いた手法の違いについて述べた.
石油精製プラント内では,異常音の取得が困難であるため正常音のみを用いて異常を検出する
教師なし学習の手法が適していることを述べた.
次に,\refsec{sec:related_work_mobile}では,移動ロボットを用いた異常音検知に関する研究について述べた.
移動ロボットの先行研究を2つ紹介し,それらの手法の課題を述べた.
一方の研究では,異常音の抽出をヒューリスティックに設計したフィルタを通すことで行っているが,
異常音の特徴が多様な場合には適用が難しい.
もう一方の研究では,点検区間をグリッドに分割し,各グリッドにおける異常音の有無を判定する手法を提案しているが,
グリッドの解像度でしか異常音の位置を特定することができない.

以上に挙げた,先行研究の課題を解決するため,本研究では,
異常音の教師データが不要な異常の位置特定を可能にする手法の構築を目指す.

% \begin{table}[htbp]
%   \centering
%   \caption{教師あり学習とグリッド分割による異常音検知の比較}
%   \label{tab:comparison}
%   \begin{tabular}{|c|c|c|}
%   \hline
%    & 異常の位置特定 & 正常音のみによる訓練 \\ \hline
%   教師あり学習 & 可能 & 不可能 \\ \hline
%   グリッド分割による異常音検知 & 不可能 & 可能 \\ \hline
%   \end{tabular}
% \end{table}


\end{document}
