\documentclass[../main]{subfiles}

\graphicspath{{../figures/}}

\begin{document}

\section{屋内実験}
\label{sec:vexp_spectral-reflectance}

\subsection{実験環境}
\label{subsec:vexp_ref_environmet}
\section{セットアップ}

全体のセットアップは図3に示されている。シュミレートされた点検ロボットにはShure SM11マイクロフォンが搭載され、Zoom P4とペアリングして48 kHzでオーディオデータを取得した。位置データはARマーカーとLogicool C925eを使用して収集され、全体のセットアップを俯瞰して記録された。

油精製所に似た騒がしい環境をシミュレートするために、タミヤ教育用コンストラクションシリーズで提供されるさまざまなサイズの小型ギアボックスを使用した(図7参照)。これらのギアボックスは構造が比較的単純で、許容範囲が広いため、異常音は傾けることで再現された。この操作により、ギアが互いに擦れ合い、研磨音が発生した。

この環境でデータを収集するために合計6回の走行を実施した。4回はすべてのギアボックスが正常に作動している状態で行われ、正常サンプルとしてラベル付けされた。2回は1つのギアボックスが異常状態で作動している状態で行われ、異常サンプルとしてラベル付けされた。合計で16868個の正常サンプルと8434個の異常サンプルが得られた。

Melスペクトログラムへの変換は97のMelフィルタバンクを用いて行われた。3つの正常な走行はトレーニングに使用され、1つの正常な走行と2つの異常な走行はテストに使用された。正常状態の音空間マップを学習するために、6層のデコーダ型ニューラルネットワークを使用した。異常音検知の閾値は手動で設定された。


\subsection{実験結果}
まず,正常音のマッピングによる経路内における異常判別結果を示す.
図に示すように,正常音のデータに対しては,経路内のすべての座標において,観測音と正常音の差分エネルギーが経路を通じて極めて小さいことが確認できる.
一方で異常音源の存在する環境下では図5に示すように,異常音源を設置した付近の経路においてエネルギーが急激に増加していることが確認できる.これらから,
ニューラルネットワークを用いた正常音のマッピングによる異常の検出が可能であることが確かめられた.

次に,異常音源の座標推定結果を示す.図6に示すように,異常音源の座標を精度よく推定できていることが確認できる.
\subsection{考察}


\label{subsec:vexp_ref_result}

\end{document}
