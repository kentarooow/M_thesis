\documentclass{/workdir/classes/summary}

\graphicspath{{figures/}}

% タイトルの項目
\title{和文題目}
\studentid{00-000000}
\author{◯◯ ◯◯}
\supervisor{◯◯ ◯◯ 教授}
\abst{
  This paper gives instructions on writing and formatting abstract papers of master thesis and reports for exercise courses at the Department of Human and Engineered Environmental Studies.
  The English abstract section of the paper should be located below title/author section and written using 10pt roman font.
  It should summarize the contents of the paper and should contain 70 to 150 words.
  Also, the abstract must be followed by a list of keywords that effectively describe domain of your work.
}

\begin{document}
\maketitle

\section{序論}
本文は2段組みとする.
改行を入れて読みやすく書く.

引用例は\cite{Schlick1994}\cite{Chikushi2020}このようになる.

本文は2段組みとする.
本文は2段組みとする.
本文は2段組みとする.
本文は2段組みとする.
本文は2段組みとする.
本文は2段組みとする.
本文は2段組みとする.
本文は2段組みとする.
本文は2段組みとする.
本文は2段組みとする.
本文は2段組みとする.
本文は2段組みとする.
本文は2段組みとする.
本文は2段組みとする.

本文は2段組みとする.
本文は2段組みとする.
本文は2段組みとする.
本文は2段組みとする.
本文は2段組みとする.
本文は2段組みとする.
本文は2段組みとする.
本文は2段組みとする.
本文は2段組みとする.
本文は2段組みとする.
本文は2段組みとする.
本文は2段組みとする.
本文は2段組みとする.
本文は2段組みとする.

本文は2段組みとする.
本文は2段組みとする.
本文は2段組みとする.
本文は2段組みとする.
本文は2段組みとする.
本文は2段組みとする.
本文は2段組みとする.
本文は2段組みとする.
本文は2段組みとする.
本文は2段組みとする.
本文は2段組みとする.
本文は2段組みとする.
本文は2段組みとする.
本文は2段組みとする.

\section{提案手法}
\subsection{対象とするシステムの概要}
次のようにすべての図表は必ず本文中で引用して説明する.
\reffig{fig:logo}はロゴである.

本文は2段組みとする.
本文は2段組みとする.
本文は2段組みとする.
本文は2段組みとする.
本文は2段組みとする.
本文は2段組みとする.
本文は2段組みとする.
本文は2段組みとする.
本文は2段組みとする.
本文は2段組みとする.
本文は2段組みとする.
本文は2段組みとする.
本文は2段組みとする.
本文は2段組みとする.

本文は2段組みとする.
本文は2段組みとする.
本文は2段組みとする.
本文は2段組みとする.
本文は2段組みとする.
本文は2段組みとする.
本文は2段組みとする.
本文は2段組みとする.
本文は2段組みとする.
本文は2段組みとする.
本文は2段組みとする.
本文は2段組みとする.
本文は2段組みとする.
本文は2段組みとする.

本文は2段組みとする.
本文は2段組みとする.
本文は2段組みとする.
本文は2段組みとする.
本文は2段組みとする.
本文は2段組みとする.
本文は2段組みとする.
本文は2段組みとする.
本文は2段組みとする.
本文は2段組みとする.
本文は2段組みとする.
本文は2段組みとする.
本文は2段組みとする.
本文は2段組みとする.

本文は2段組みとする.
本文は2段組みとする.
本文は2段組みとする.
本文は2段組みとする.
本文は2段組みとする.
本文は2段組みとする.
本文は2段組みとする.
本文は2段組みとする.
本文は2段組みとする.
本文は2段組みとする.
本文は2段組みとする.
本文は2段組みとする.
本文は2段組みとする.
本文は2段組みとする.

\begin{equation}
  r = \left\{
    \begin{array}{cc}
      10   & \text{if success} \\
      - 10 & \text{if fail}    \\
      - d  & \text{else}
    \end{array}
  \right. \text{.}
\end{equation}

\begin{figure}[tb]
  \centering
  \includegraphics[keepaspectratio, width=0.8\linewidth]{utlogo.pdf}
  \caption{ロゴ}
  \label{fig:logo}
\end{figure}

\subsection{提案手法}
本文は2段組みとする.
本文は2段組みとする.
本文は2段組みとする.
本文は2段組みとする.
本文は2段組みとする.
本文は2段組みとする.
本文は2段組みとする.
本文は2段組みとする.
本文は2段組みとする.
本文は2段組みとする.
本文は2段組みとする.
本文は2段組みとする.
本文は2段組みとする.
本文は2段組みとする.

本文は2段組みとする.
本文は2段組みとする.
本文は2段組みとする.
本文は2段組みとする.
本文は2段組みとする.
本文は2段組みとする.
本文は2段組みとする.
本文は2段組みとする.
本文は2段組みとする.
本文は2段組みとする.
本文は2段組みとする.
本文は2段組みとする.
本文は2段組みとする.
本文は2段組みとする.

本文は2段組みとする.
本文は2段組みとする.
本文は2段組みとする.
本文は2段組みとする.
本文は2段組みとする.
本文は2段組みとする.
本文は2段組みとする.
本文は2段組みとする.
本文は2段組みとする.
本文は2段組みとする.
本文は2段組みとする.
本文は2段組みとする.
本文は2段組みとする.
本文は2段組みとする.

本文は2段組みとする.
本文は2段組みとする.
本文は2段組みとする.
本文は2段組みとする.
本文は2段組みとする.
本文は2段組みとする.
本文は2段組みとする.
本文は2段組みとする.
本文は2段組みとする.
本文は2段組みとする.
本文は2段組みとする.
本文は2段組みとする.
本文は2段組みとする.
本文は2段組みとする.

\begin{figure}[t]
  \centering
  \includegraphics[keepaspectratio, width=0.5\linewidth]{chap1/cone_penetrometer.pdf}
  \caption{コーンペネトロメータ}
  \label{fig:cone_penetrometer}
\end{figure}

% textlint-disable
\begin{table}[t]
  \caption{走行に必要なコーン指数}
  \label{tab:traffic_cone_index}
  \centering
  \begin{tabular}{cc}
    \toprule
    建設機械の種類                      & コーン指数[\si{\kN/\m^2}] \\
    \midrule
    超湿地ブルドーザー                    & 200以上                \\
    湿地ブルドーザー                     & 300以上                \\
    普通ブルドーザー(\SI{15}{\tonne}級程度) & 500以上                \\
    普通ブルドーザー(\SI{21}{\tonne}級程度) & 700以上                \\
    スクレープドーザ                     & 600以上                \\
                                 & (超湿地型は400以上)         \\
    被けん引式スクレーパ(小型)               & 700以上                \\
    自走式スクレーパ(小型)                 & 1,000以上              \\
    ダンプトラック                      & 1,200以上              \\
    \bottomrule
  \end{tabular}
  \vspace{\zh}
\end{table}
% textlint-enable

\section{検証実験}
本文は2段組みとする.
本文は2段組みとする.
本文は2段組みとする.
本文は2段組みとする.
本文は2段組みとする.
本文は2段組みとする.
本文は2段組みとする.
本文は2段組みとする.
本文は2段組みとする.
本文は2段組みとする.
本文は2段組みとする.
本文は2段組みとする.
本文は2段組みとする.
本文は2段組みとする.

本文は2段組みとする.
本文は2段組みとする.
本文は2段組みとする.
本文は2段組みとする.
本文は2段組みとする.
本文は2段組みとする.
本文は2段組みとする.
本文は2段組みとする.
本文は2段組みとする.
本文は2段組みとする.
本文は2段組みとする.
本文は2段組みとする.
本文は2段組みとする.
本文は2段組みとする.

本文は2段組みとする.
本文は2段組みとする.
本文は2段組みとする.
本文は2段組みとする.
本文は2段組みとする.
本文は2段組みとする.
本文は2段組みとする.
本文は2段組みとする.
本文は2段組みとする.
本文は2段組みとする.
本文は2段組みとする.
本文は2段組みとする.
本文は2段組みとする.
本文は2段組みとする.

本文は2段組みとする.
本文は2段組みとする.
本文は2段組みとする.
本文は2段組みとする.
本文は2段組みとする.
本文は2段組みとする.
本文は2段組みとする.
本文は2段組みとする.
本文は2段組みとする.
本文は2段組みとする.
本文は2段組みとする.
本文は2段組みとする.
本文は2段組みとする.
本文は2段組みとする.

\section{実験結果・考察}
本文は2段組みとする.
本文は2段組みとする.
本文は2段組みとする.
本文は2段組みとする.
本文は2段組みとする.
本文は2段組みとする.
本文は2段組みとする.
本文は2段組みとする.
本文は2段組みとする.
本文は2段組みとする.
本文は2段組みとする.
本文は2段組みとする.
本文は2段組みとする.
本文は2段組みとする.

本文は2段組みとする.
本文は2段組みとする.
本文は2段組みとする.
本文は2段組みとする.
本文は2段組みとする.
本文は2段組みとする.
本文は2段組みとする.
本文は2段組みとする.
本文は2段組みとする.
本文は2段組みとする.
本文は2段組みとする.
本文は2段組みとする.
本文は2段組みとする.
本文は2段組みとする.

本文は2段組みとする.
本文は2段組みとする.
本文は2段組みとする.
本文は2段組みとする.
本文は2段組みとする.
本文は2段組みとする.
本文は2段組みとする.
本文は2段組みとする.
本文は2段組みとする.
本文は2段組みとする.
本文は2段組みとする.
本文は2段組みとする.
本文は2段組みとする.
本文は2段組みとする.

本文は2段組みとする.
本文は2段組みとする.
本文は2段組みとする.
本文は2段組みとする.
本文は2段組みとする.
本文は2段組みとする.
本文は2段組みとする.
本文は2段組みとする.
本文は2段組みとする.
本文は2段組みとする.
本文は2段組みとする.
本文は2段組みとする.
本文は2段組みとする.
本文は2段組みとする.

本文は2段組みとする.
本文は2段組みとする.
本文は2段組みとする.
本文は2段組みとする.
本文は2段組みとする.
本文は2段組みとする.
本文は2段組みとする.
本文は2段組みとする.
本文は2段組みとする.
本文は2段組みとする.
本文は2段組みとする.
本文は2段組みとする.
本文は2段組みとする.
本文は2段組みとする.

\section{結論}
本文は2段組みとする.
本文は2段組みとする.
本文は2段組みとする.
本文は2段組みとする.
本文は2段組みとする.
本文は2段組みとする.
本文は2段組みとする.
本文は2段組みとする.
本文は2段組みとする.
本文は2段組みとする.
本文は2段組みとする.
本文は2段組みとする.
本文は2段組みとする.
本文は2段組みとする.

本文は2段組みとする.
本文は2段組みとする.
本文は2段組みとする.
本文は2段組みとする.
本文は2段組みとする.
本文は2段組みとする.
本文は2段組みとする.
本文は2段組みとする.
本文は2段組みとする.
本文は2段組みとする.
本文は2段組みとする.
本文は2段組みとする.
本文は2段組みとする.
本文は2段組みとする.

本文は2段組みとする.
本文は2段組みとする.
本文は2段組みとする.
本文は2段組みとする.
本文は2段組みとする.
本文は2段組みとする.
本文は2段組みとする.
本文は2段組みとする.
本文は2段組みとする.
本文は2段組みとする.
本文は2段組みとする.
本文は2段組みとする.
本文は2段組みとする.
本文は2段組みとする.

\bibliographystyle{summary}
\bibliography{sections/reference,sections/reference_web}
\end{document}
